\section{Workshop Experience}
So far in this module I have completed the wood and plastics workshops. Both have been useful experiences and I have learned a lot about working quickly making best use of the power tools available, having in the past mostly used hand tools.

In the wood workshop I gained some valuable experience with different woods. Working with the pine with hand tools was similar to the woodwork I was used to. The loose grain of the wood meant that it chipped or tore easily when going the wrong direction along the wood's grain, as the grain usually did not perfectly align with the edge of the piece.

The plastics workshop was more of a new experience; I had worked with acrylic before, but never with multiple sheets glued together. It proved to be a very different experience to what I expected. Acrylic has a low toughness, and in the past my tools had evidently not been sharp enough to cut it, meaning it cracked easily and frequently. With properly maintained tools, and especially with multiple layers laminated together, the acrylic becomes much easier to work with.

While the plastics workshop deals only with acrylic, I wanted to research some other polymers, and I needed a context to relate them to. Something I have been very enthusiastic about in the past is hammock camping. When a hammock is tied between two trees it can put the rope in very high tension, which requires materials with a certain set of properties. I decided to investigate these properties and the materials which we use for ropes.