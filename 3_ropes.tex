\section{Ropes used for hammock suspension}
\subsection{Nylon and polyester webbing}
A common material for hammock suspension is fabric strips woven from nylon or polyester, known as webbing. It is used mainly because it is cheap, readily available and due to its width (\SIrange{25}{50}{\milli\metre}), which means it is unlikely to damage trees.

One problem that webbing exhibits is stretching when weight is applied. This stretching is greater in nylon webbing---20-30\% as opposed to polyester's 15-25\% \pcite{tie_down_faq}, so polyester webbing is more often used. This stretch can result in a sub-optimal ''hang`` for the camper, as the strap length increases overnight.

A fairly typical price for \SI{25}{\milli\metre} polyester webbing is £0.50 per metre. This makes it very cheap for hammock campers to buy, and its \SI{1500}{\kilo\gram} breaking strength \pcite{25mm_poly_webbing} means it is more than strong enough to withstand normal usage. At \SI{25}{\milli\metre} wide and \SI{1.25}{\milli\metre} thick, a \SI{1500}{\kilo\gram} mass would apply a strain of $1500 \times 9.81 \div (\num{25e-3} \times \num{1.25e-3}) = \SI{470}{\mega\pascal}$. This is far higher than the expected strains calculated in section \ref{sec:mechanics}, so this webbing is far stronger than needed for hammock suspension.

\subsection{High modulus polyethylene ropes}
However, such webbing is bulky, making it difficult to carry when hiking. A more lightweight, smaller alternative is needed for some campers. They usually turn to high-modulus polyethylene ropes, marketed as Dyneema or Spectra, which also solve the problem of stretching.

Ultra-high molecular weight polyethylene is the main material I will investigate in this report. Higher priced but stronger than nylon webbing, woven polyethylene ropes tend to be marketed at sailing enthusiasts rather than as general purpose cords. They are prized among those who use them for their low weight, low bulk and high strength.