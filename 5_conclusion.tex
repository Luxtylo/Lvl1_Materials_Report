\section{Conclusion}
% Relate back to hammock ropes
% Talk about how future advances could help properties
The stiffness of UHMWPE is one factor which draws hammock campers to it. The ultimate tensile strength only needs to be sufficiently high, but a low modulus can significantly affect how the hammock is hung.

This is because, when weight is placed on the hammock, a low modulus rope will stretch more than one with a higher modulus. The angle between the rope and the horizontal is affected by this---with enough extension of the rope it can completely change how comfortable the hammock is.

The high modulus of the fibres - \SIrange{109}{132}{\giga\pascal} \pcite{dyneema_factsheet} - used in ropes typically used by hammock campers (for example Kingfisher \SI{3}{\milli\metre} braided dyneema \pcite{3mm_dyneema}) combined with their low mass and bulk makes them ideal.

Estimating a usage of around \SI{8}{\metre} for a typical hammock user, and taking a weight for \SI{3}{\milli\metre} dyneema of \SI{5}{\gram\per\metre}, we find that the weight of a typical quantity of dyneema is \SI{50}{\gram}--a negligible value compared to the weight of the fabric of a hammock.